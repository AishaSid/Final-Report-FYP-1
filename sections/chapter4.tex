\chapter{Implementation and Testing [UPTO THE CURRENT ITERATION ONLY]}

Give a general description of the functionality, context, and design of your project.
Provide any background information if necessary.

\section{Algorithm Design}
Mention the algorithm(s) used in your project to get the work done with regards to major modules. Provide a pseudocode explanation regarding the functioning of the core features. Following are few examples of algorithms/pseudocode. \\

Example:
\begin{figure}[H] % Use 'H' to place the figure exactly here
    \centering
    \includegraphics[width=\textwidth]{images/algoexample.png}
    \caption{Example Of Algorithm Design}
    \label{fig:usecase3}
\end{figure}




\section{External APIs/SDKs}
Describe the third-party APIs/SDKs used in the project implementation in the following table. Few examples of APIs are provided in the table.

% Please add the following required packages to your document preamble:
% \usepackage{graphicx}
\begin{table}[!h]
\resizebox{\textwidth}{!}{%
\begin{tabular}{|l|l|l|l|}
\hline
API   and version & Description  & Purpose of   usage & API endpoint/function/class used \\ \hline
Stripe (version 2020-08-27) & Credit Card payment integration & Sandbox used orders payment & stripe.paymentMethods.create \\ \hline
Cloudinary & Image and Video management & Uploading Product Images & https://api.cloudinary.com/v1 \\ \hline
\end{tabular}%
}
\end{table}



\section{Testing Details}
Once the system has been successfully developed, testing has to be performed to ensure that the system working as intended. 

\subsection{Unit Testing}
 Each unit test is designed to test a specific function or method independently from other components, helping to identify issues directly related to the functionality being tested. 

Following is the example of Unit testing:

\begin{figure}[H] % Use 'H' to place the figure exactly here
    \centering
    \includegraphics[width=\textwidth]{images/unittesting.png}
    \caption{Example for Unit Testing}
    \label{fig:usecase3}
\end{figure}


