\chapter{Implementation and Testing}
\label{ch:implementation-testing}

This chapter describes the current implementation status of the CoWriteIA system and the testing strategies applied to validate its functionality. The focus of this phase was to implement the core backend services, major APIs, and conduct rigorous black box testing to ensure that all user-facing functionalities behave correctly according to the system requirements.

The implementation described in this chapter reflects the progress achieved up to the current iteration of the project.

\section{Algorithm Design}
\label{sec:algorithm-design}

The system implements a Retrieval-Augmented Generation (RAG) pipeline to enable intelligent document-based responses. The algorithm focuses only on the core functional requirements: document processing, semantic indexing, context retrieval, and AI response generation.

The CoWriteIA implementation in this repository realizes a semantic indexing + RAG (retrieval-augmented generation) pipeline. The implemented components (and their primary locations in the codebase) are:

\begin{itemize}
    \item Document chunking and preprocessing
    \begin{itemize}
        \item \texttt{backend/app/services/text\_extraction\_service.py :: TextExtractionService.chunk\_text / \_create\_text\_chunk}
    \end{itemize}
    \item Embedding generation and vector storage
    \begin{itemize}
        \item \texttt{backend/app/services/embedding\_service.py :: EmbeddingService.generate\_single\_embedding, generate\_embeddings, store\_text\_chunks\_with\_embeddings}
        \item \texttt{backend/app/services/chroma\_service.py :: ChromaService.add\_embeddings, search\_similar}
        \item \texttt{backend/app/repositories/text\_chunk\_repository.py :: update\_embedding, get\_chunks\_without\_embeddings, vector\_search / vector\_search\_chroma}
    \end{itemize}
    \item Context retrieval (semantic similarity)
    \begin{itemize}
        \item \texttt{backend/app/services/rag\_context\_service.py :: RAGContextService.assemble\_context}
        \item \texttt{backend/app/services/search\_service.py :: SearchService.semantic\_search, find\_similar\_content}
    \end{itemize}
    \item AI response generation (RAG orchestration and assistant)
    \begin{itemize}
        \item \texttt{backend/app/services/ai\_chat\_service.py :: AIChatService.chat}
        \item \texttt{backend/app/services/copilot\_service.py :: CopilotService.generate\_suggestion}
    \end{itemize}
\end{itemize}

The following subsections contain concise algorithm descriptions derived from the actual implementation and precise pseudocode that maps to the repository functions.

\subsection{High-level Flow}
\begin{enumerate}
    \item User uploads file via endpoints in \texttt{backend/app/api/v1/endpoints/files.py}.
    \item File processing triggers text extraction and chunking:
    \begin{itemize}
        \item \texttt{TextExtractionService.extract\_text\_from\_file(...)} and \texttt{TextExtractionService.chunk\_text(...)}
        \item Chunks are created as \texttt{TextChunk} models and persisted via \texttt{TextChunkRepository}.
    \end{itemize}
    \item Embeddings are generated for chunks:
    \begin{itemize}
        \item \texttt{EmbeddingService.generate\_embeddings(...)} / \texttt{generate\_single\_embedding(...)}
        \item Embeddings saved both in chunk metadata (MongoDB) and in ChromaDB via \texttt{ChromaService} when available.
    \end{itemize}
    \item Semantic search / RAG:
    \begin{itemize}
        \item \texttt{RAGContextService.assemble\_context(...)} calls embedding/search services to retrieve top-$k$ chunks.
        \item \texttt{SearchService} handles hybrid/fallback logic (ChromaDB preferred, fallback to MongoDB vector search).
    \end{itemize}
    \item LLM generation:
    \begin{itemize}
        \item \texttt{AIChatService.chat(...)} or \texttt{CopilotService.generate\_suggestion(...)} build a prompt from the retrieved context and call the configured LLM (\texttt{GroqService} or other).
    \end{itemize}
    \item Results returned to the client; metadata and analytics stored as needed.
\end{enumerate}

\subsection*{Example short sequence (mapping code $\rightarrow$ runtime)}
\begin{enumerate}
    \item User uploads file $\rightarrow$ API endpoint in \texttt{backend/app/api/v1/endpoints/files.py} triggers \texttt{process\_uploaded\_file}.
    \item \texttt{process\_uploaded\_file} calls \texttt{TextExtractionService.chunk\_text} to create chunks.
    \item Indexing background task (or immediate synchronous call) calls \texttt{EmbeddingService.generate\_embeddings} for chunks.
    \item \texttt{EmbeddingService} stores embeddings via \texttt{TextChunkRepository} and \texttt{ChromaService}.
    \item Later, user sends search/query $\rightarrow$ \texttt{RAGContextService.assemble\_context} retrieves top chunks using \texttt{ChromaService} or \texttt{SearchService}.
    \item \texttt{AIChatService.chat} builds prompt with context and calls \texttt{GroqService} (LLM); response persisted and returned.
\end{enumerate}

\subsection{Pseudocode: Document Chunking and Preprocessing}

The system splits extracted text into sentence-aware chunks within a target size and optionally applies overlap. Each chunk is materialized using the implementation in \texttt{backend/app/services/text\_extraction\_service.py}.

\begin{algorithm}[H]
\caption{ChunkTextAndCreateChunks}
\begin{algorithmic}[1]
\Require \textit{file\_id}, \textit{project\_id}, \textit{full\_text}, \textit{max\_chunk\_size}, \textit{overlap} (optional), \textit{start\_chunk\_index} (default 0)
\Ensure List of \texttt{TextChunk} models \textit{chunks}
\State sentences $\leftarrow$ \texttt{TextExtractionService.\_split\_into\_sentences}(full\_text)
\State chunks $\leftarrow$ [\,]
\State current\_chunk\_sentences $\leftarrow$ [\,];\; current\_word\_count $\leftarrow$ 0;\; chunk\_index $\leftarrow$ start\_chunk\_index
\For{each sentence $s$ in sentences}
    \State sentence\_words $\leftarrow$ \texttt{count\_words}($s$)
    \If{current\_word\_count + sentence\_words $>$ max\_chunk\_size \textbf{and} current\_chunk\_sentences not empty}
        \State chunk\_text $\leftarrow$ \texttt{join}(current\_chunk\_sentences, " ")
        \State start\_pos $\leftarrow$ compute absolute start in \textit{full\_text}
        \State chunk\_model $\leftarrow$ \texttt{TextExtractionService.\_create\_text\_chunk}(file\_id, project\_id, chunk\_text, chunk\_index, start\_pos, current\_word\_count)
        \State append chunk\_model to chunks
        \State chunk\_index $\leftarrow$ chunk\_index + 1
        \If{overlap configured}
            \State current\_chunk\_sentences $\leftarrow$ take last \textit{overlap} words/sentences of current\_chunk\_sentences
            \State current\_word\_count $\leftarrow$ \texttt{count\_words}(current\_chunk\_sentences)
        \Else
            \State current\_chunk\_sentences $\leftarrow$ [\,];\; current\_word\_count $\leftarrow$ 0
        \EndIf
    \EndIf
    \State append $s$ to current\_chunk\_sentences;\; current\_word\_count $\leftarrow$ current\_word\_count + sentence\_words
\EndFor
\If{current\_chunk\_sentences not empty}
    \State chunk\_text $\leftarrow$ \texttt{join}(current\_chunk\_sentences, " ")
    \State start\_pos $\leftarrow$ compute absolute start in \textit{full\_text}
    \State chunk\_model $\leftarrow$ \texttt{TextExtractionService.\_create\_text\_chunk}(file\_id, project\_id, chunk\_text, chunk\_index, start\_pos, current\_word\_count)
    \State append chunk\_model to chunks
\EndIf
\State \Return chunks
\end{algorithmic}
\end{algorithm}
\FloatBarrier

\textbf{Notes:}
\begin{itemize}
    \item Uses sentence-aware splitting and target size with optional overlap per \texttt{TextExtractionService}.
    \item Returns \texttt{TextChunk} Pydantic models; persistence occurs later (e.g., in \texttt{store\_text\_chunks\_with\_embeddings}).
\end{itemize}

\subsection{Embedding Generation and Storage}

This stage encodes chunked text into dense vectors and persists both the metadata and vectors. It maps to \texttt{EmbeddingService} and \texttt{ChromaService} usage in the codebase.

\begin{algorithm}[H]
\caption{GenerateAndPersistEmbeddings}
\begin{algorithmic}[1]
\Require \textit{project\_id}, \textit{file\_id}, \textit{list\_of\_chunks} (content strings)
\Ensure List of created/updated \texttt{TextChunk} records with embeddings
\State chunk\_texts $\leftarrow$ [\,\texttt{chunk.content} for \textit{chunk} in \textit{list\_of\_chunks}\,]
\State embeddings $\leftarrow$ \texttt{EmbeddingService.generate\_embeddings}(chunk\_texts)
\Comment{Runs threadpool synchronous encoding; model loads lazily}
\State created\_chunks $\leftarrow$ [\,]
\For{each $(chunk\_info, embedding)$ in \texttt{zip}(list\_of\_chunks, embeddings)}
    \State chunk\_model $\leftarrow$ \texttt{build TextChunk}(\,
        file\_id $=$ file\_id,\; project\_id $=$ project\_id,\;\
        content $=$ chunk\_info.content,\; start\_position $= \dots$,\; end\_position $= \dots$,\;\
        chunk\_index $=$ chunk\_info.chunk\_index,\; word\_count $= \dots$,\;\
        embedding\_vector $=$ embedding\,)
    \State saved\_chunk $\leftarrow$ \texttt{TextChunkRepository.create}(chunk\_model)
    \Comment{Persist metadata in MongoDB}
    \State append saved\_chunk to created\_chunks
\EndFor
\If{\texttt{ChromaService} available}
    \State chroma\_entries $\leftarrow$ map \textit{created\_chunks} to \{\texttt{id}, \texttt{embedding}, \texttt{metadata}\}
    \State \texttt{ChromaService.add\_embeddings}(project\_id, chroma\_entries)
\Else
    \State Optionally store embeddings only within MongoDB fields
\EndIf
\State \Return created\_chunks
\end{algorithmic}
\end{algorithm}
\FloatBarrier

\textbf{Implementation highlights:}
\begin{itemize}
    \item \textbf{Embedding model:} \texttt{sentence-transformers/all-MiniLM-L6-v2} (see embedding service and verification scripts).
    \item \textbf{Dual storage:} Metadata in MongoDB \texttt{text\_chunk} documents; vectors in ChromaDB via \texttt{backend/app/services/chroma\_service.py}.
    \item \textbf{Batch background tasks:} Celery jobs generate missing embeddings in \texttt{backend/app/tasks/indexing.py} (\texttt{batch\_generate\_embeddings\_task}).
\end{itemize}

\subsection{Pseudocode: Context-aware Retrieval (RAG)}

This step assembles a rich context from semantically similar chunks and optional entity/scene/relationship metadata.

\begin{algorithm}[H]
\caption{AssembleRAGContext}
\begin{algorithmic}[1]
\Require query \textit{Q}, \textit{project\_id}, optional \textit{file\_id}, max\_chunks \textit{k}
\Ensure \textit{context\_payload}: list of chunks, entities, scenes, relationships
\State query\_embedding $\leftarrow$ \texttt{EmbeddingService.generate\_single\_embedding}(Q)
\If{\texttt{ChromaService} available}
    \State vector\_hits $\leftarrow$ \texttt{ChromaService.search\_similar}(project\_id, query\_embedding, \texttt{n\_results} $=$ k)
\Else
    \State vector\_hits $\leftarrow$ \texttt{TextChunkRepository.vector\_search}(project\_id, query\_embedding, \texttt{limit} $=$ k)
\EndIf
\State top\_chunks $\leftarrow$ convert \textit{vector\_hits} to chunk payloads (\texttt{id}, \texttt{content}, \texttt{score}, \texttt{file\_id}, \texttt{positions})
\If{include\_entities}
    \State entities $\leftarrow$ \texttt{EntityRepository.get\_by\_project}(project\_id)
    \State entity\_matches $\leftarrow$ identify entities mentioned in \textit{top\_chunks} (\texttt{EntityExtractionService} or string matching)
\EndIf
\If{include\_scenes}
    \State scenes $\leftarrow$ \texttt{SceneRepository.find\_by\_project}(project\_id)
    \State mapped\_scenes $\leftarrow$ map chunks to scenes when positions overlap or proximity matches
\EndIf
\If{include\_relationships}
    \State relationships $\leftarrow$ \texttt{RelationshipRepository.find\_by\_project}(project\_id)
\EndIf
\State context\_payload $\leftarrow$ \{\texttt{"chunks"}: top\_chunks, \texttt{"entities"}: entity\_matches, \texttt{"scenes"}: mapped\_scenes, \texttt{"relationships"}: relationships (filtered)\}
\State \Return context\_payload
\end{algorithmic}
\end{algorithm}
\FloatBarrier

\textbf{Notes:}
\begin{itemize}
    \item \texttt{RAGContextService.assemble\_context} collects chunk-level results, enriches them with entity and scene metadata, and returns a compact context mapping suitable for prompt building (see \texttt{backend/app/services/rag\_context\_service.py}).
    \item \texttt{SearchService} implements hybrid logic and fallback ordering (Chroma $\rightarrow$ MongoDB \texttt{vectorSearch}).
\end{itemize}

\subsection{Pseudocode: AI Response Generation (RAG + LLM Call)}

This step orchestrates context assembly, prompt construction, LLM invocation, and persistence. It corresponds to \texttt{AIChatService.chat} and \texttt{CopilotService.generate\_suggestion}.

\begin{algorithm}[H]
\caption{GenerateAIResponseWithContext}
\begin{algorithmic}[1]
\Require \textit{user\_message}, \textit{project\_id}, optional \textit{file\_id}, \textit{user\_id}, \textit{max\_context\_chunks}
\Ensure \textit{assistant\_message}
\State \textbf{Step A: Assemble RAG context}
\State context $\leftarrow$ \texttt{RAGContextService.assemble\_context}(\,\texttt{query} $=$ user\_message, \texttt{project\_id} $=$ project\_id, \texttt{file\_id} $=$ file\_id, \texttt{max\_chunks} $=$ max\_context\_chunks\,)
\State \textbf{Step B: Format context into LLM prompt}
\State formatted\_context $\leftarrow$ \texttt{RAGContextService.format\_context\_for\_llm}(context)
\State system\_prompt $\leftarrow$ \texttt{AIChatService.SYSTEM\_PROMPT}
\State conversation\_history $\leftarrow$ \texttt{ConversationRepository.get\_latest\_messages}(conversation\_id or new)
\State llm\_messages $\leftarrow$ \texttt{build\_messages\_list}(system\_prompt, conversation\_history, user\_message, formatted\_context)
\Comment{system message, recent assistant/user turns, current user message, appended context}
\State \textbf{Step C: Call LLM}
\State assistant\_content $\leftarrow$ \texttt{GroqService.generate}(\texttt{messages} $=$ llm\_messages, \texttt{max\_tokens} $= \dots$, \texttt{model} $= \dots$)
\State \textbf{Step D: Postprocess and store}
\State cleaned $\leftarrow$ \texttt{clean\_text}(assistant\_content) \Comment{e.g., \texttt{CopilotService.\_clean\_suggestion}}
\State \texttt{MessageRepository.save}(role $=$ \texttt{assistant}, content $=$ cleaned, context\_used $=$ context)
\State update conversation metadata (message count, token usage, timestamps)
\State \Return cleaned
\end{algorithmic}
\end{algorithm}
\FloatBarrier

\textbf{Implementation notes:}
\begin{itemize}
    \item \texttt{AIChatService.chat} orchestrates assembling context, formatting prompts, invoking the LLM, and persisting messages.
    \item \texttt{CopilotService.generate\_suggestion} targets inline assistance: it infers writing style from RAG results, calls the LLM, then returns a cleaned suggestion and the context used (see \texttt{backend/app/services/copilot\_service.py}).
\end{itemize}

\subsection*{Complexity, Reliability, and Practical Concerns}
\label{subsec:complexity-reliability-practical}

\begin{itemize}
    \item \textbf{Chunking complexity:} $O(n)$ in the number of sentences per document. Overlap increases effective storage but preserves local context for downstream retrieval.
    \item \textbf{Embedding generation:} Dominated by model inference cost per chunk. Batch generation reduces overhead (indexing background tasks \texttt{batch\_generate\_missing\_embeddings}).
    \item \textbf{Vector search:} ChromaDB uses efficient ANN (HNSW) with sub-linear query cost; fallback MongoDB vector aggregation may be slower for large collections.
    \item \textbf{Fault tolerance:}
    \begin{itemize}
        \item Embedding/model calls may fail due to transient network or provider issues; services use \texttt{try/catch} with sensible fallbacks (see embedding service and verify scripts).
        \item RAG assembly respects available services and degrades gracefully (ChromaDB optional; code falls back to MongoDB vector search when unavailable).
    \end{itemize}
    \item \textbf{Reindexing \& incremental updates:} \texttt{TextExtractionService.reprocess\_file} deletes existing chunks and marks the file ready for reprocessing; batch background tasks handle any missing embeddings.
\end{itemize}

\begin{algorithm}[H]
\caption{ReprocessFileAndIncrementalUpdate}
\begin{algorithmic}[1]
\Require \textit{file\_id}, \textit{project\_id}
\Ensure File is reindexed and embeddings regenerated as needed
\State \texttt{TextChunkRepository.delete\_by\_file}(file\_id)
\State \texttt{FileRepository.mark\_ready\_for\_processing}(file\_id)
\State full\_text $\leftarrow$ \texttt{TextExtractionService.extract\_text\_from\_file}(file\_id)
\State chunks $\leftarrow$ \texttt{TextExtractionService.chunk\_text}(project\_id, file\_id, full\_text, \texttt{max\_chunk\_size}, \texttt{overlap})
\State \texttt{TextChunkRepository.bulk\_insert}(chunks)
\State enqueue \texttt{batch\_generate\_missing\_embeddings}(project\_id, file\_id)
\end{algorithmic}
\end{algorithm}
\FloatBarrier

\subsection*{Concrete File References (Key Implementation Points)}
\begin{itemize}
    \item \textbf{Text chunking and models}
    \begin{itemize}
        \item \texttt{backend/app/services/text\_extraction\_service.py} :: \texttt{chunk\_text()}, \texttt{\_create\_text\_chunk()}, \texttt{\_split\_into\_sentences()}
        \item \texttt{backend/app/repositories/text\_chunk\_repository.py} :: \texttt{create()}, \texttt{get\_by\_file()}, \texttt{get\_chunks\_without\_embeddings()}, \texttt{update\_embedding()}
    \end{itemize}
    \item \textbf{Embeddings and vector store}
    \begin{itemize}
        \item \texttt{backend/app/services/embedding\_service.py} :: \texttt{generate\_single\_embedding()}, \texttt{generate\_embeddings()}, \texttt{store\_text\_chunks\_with\_embeddings()}, \texttt{find\_similar\_chunks()}
        \item \texttt{backend/app/services/chroma\_service.py} :: \texttt{get\_or\_create\_collection()}, \texttt{add\_embeddings()}, \texttt{search\_similar()}
        \item \texttt{backend/app/tasks/indexing.py} :: \texttt{index\_document\_task()}, \texttt{batch\_generate\_embeddings\_task()}
    \end{itemize}
    \item \textbf{RAG and search}
    \begin{itemize}
        \item \texttt{backend/app/services/rag\_context\_service.py} :: \texttt{assemble\_context()}, \texttt{format\_context\_for\_llm()}
        \item \texttt{backend/app/services/search\_service.py} :: \texttt{semantic\_search()}, \texttt{find\_similar\_content()}, hybrid logic
    \end{itemize}
    \item \textbf{LLM orchestration and assistants}
    \begin{itemize}
        \item \texttt{backend/app/services/ai\_chat\_service.py} :: \texttt{chat()}
        \item \texttt{backend/app/services/copilot\_service.py} :: \texttt{generate\_suggestion()}, \texttt{\_gather\_story\_context()}
    \end{itemize}
\end{itemize}

% See Appendix E for call signatures
See Appendix~\ref{sec:appendix-e-pseudo-api} for pseudo-API call signatures.

\section{External APIs and SDKs}
\label{sec:external-apis}

The system integrates multiple third-party APIs and SDKs to support AI capabilities, authentication, and data persistence.

\begin{table}[H]
\centering
\begin{tabular}{|p{3cm}|p{4cm}|p{4cm}|p{3cm}|}
\hline
\textbf{API / SDK} & \textbf{Description} & \textbf{Purpose of Usage} & \textbf{Endpoint/ Function} \\
\hline
OpenAI API (v1) & Large language model services & Text generation and embeddings & /v1/chat/completions \\
\hline
FastAPI & Python web framework & REST API backend & @app.get(), @app.post() \\
\hline
PostgreSQL & Relational database system & User and project data storage & psycopg2 driver \\
\hline
VectorDB (FAISS/Pinecone) & Vector similarity engine & Semantic search and retrieval & similarity\_search() \\
\hline
\end{tabular}
\caption{External APIs and SDKs Used}
\end{table}

\section{Testing Details}
\label{sec:testing-details}

Testing played a critical role in validating the correctness, reliability, and stability of the implemented system. A combination of black box testing and unit testing strategies was adopted to ensure that the system meets its functional requirements.

\subsection{Black Box Testing}
\label{subsec:blackbox-testing}

Black box testing was used to validate the external behavior of the system without reference to internal implementation details. The focus was on verifying that system features produced correct outputs when interacting through public interfaces such as API endpoints.

\subsubsection{Purpose of Black Box Testing}

The objective of this testing was to ensure that system functionality aligned with the documented functional requirements by validating response codes, output structures, and observable system behavior.

\subsubsection{Testing Environment}

Testing was conducted using an isolated staging environment that included a running API server, a connected database, and a REST API testing client. All tests were executed externally without accessing source code.

% Insert environment screenshot

\subsubsection{Functional API Coverage}

The following functional modules were tested as black box components:

\begin{itemize}
    \item Authentication services (registration, login, token validation)
    \item Project management operations
    \item File handling operations
    \item Search and retrieval services
    \item Chat and conversational APIs
\end{itemize}

% Insert API success response screenshots

\subsubsection{Workflow and Scenario Testing}

End-to-end user workflows were validated through multi-step test scenarios, including project creation, file indexing, semantic search, and AI-assisted responses.

% Insert workflow execution screenshots

\subsubsection{Negative and Security Testing}

Invalid inputs, unauthorized requests, and forbidden access attempts were tested to verify that the system safely rejected improper usage without exposing internal errors.

% Insert 401 / 403 screenshots

\subsubsection{Error and Edge Case Validation}

Boundary conditions such as missing data, empty result sets, and invalid resource requests were evaluated to ensure stable system responses.

% Insert error response screenshots

\subsubsection{Black Box Test Evidence}

All executed test cases were recorded and maintained as structured documentation.

\begin{figure}[H]
    \centering
    \includegraphics[width=0.90\textwidth]{images/tests1.png}
    \caption{BlackBox Tests fig:1}
\end{figure}

\begin{figure}[H]
    \centering
    \includegraphics[width=0.90\textwidth]{images/tests2.png}
    \caption{BlackBox Tests fig:2}
\end{figure}

\begin{figure}[H]
    \centering
    \includegraphics[width=0.90\textwidth]{images/bboxutest1.png}
    \caption{BlackBox Tests fig:3}
    \label{fig:bboxutest1}
\end{figure}

\begin{figure}[H]
    \centering
    \includegraphics[width=0.90\textwidth]{images/bboxutest2.png}
    \caption{BlackBox Tests fig:4}
\end{figure}

\begin{figure}[H]
    \centering
    \includegraphics[width=0.90\textwidth]{images/bboxutest3.png}
    \caption{BlackBox Tests fig:5}
    \label{fig:bboxutest3}
\end{figure}

\subsubsection{Black Box Testing Summary}

The system demonstrated consistent and correct behavior for valid use cases and safely handled invalid or unauthorized operations.

\subsection{Unit Testing}
\label{subsec:unit-testing}

Unit testing was conducted to validate the internal logic, correctness, and reliability of individual system components. Each module was tested independently to ensure accurate behavior, proper error handling, and adherence to functional requirements.

\subsubsection{Implemented Unit Test Coverage}

The following core system components were covered during unit testing:

\begin{itemize}
    \item Authentication services (registration, login, token handling)
    \item File processing and metadata management
    \item Search and autocomplete operations
    \item Text extraction, chunking, and incremental processing
    \item Embedding generation, statistics, and similarity scoring
    \item LLM interaction services (GroqService)
    \item Edit proposal generation and diff processing
    \item Copilot assistance (suggestions, style analysis, prompt building)
    \item RAG-based context assembly
    \item Entity extraction and validation
    \item Relationship discovery and contextual analysis
\end{itemize}

\subsubsection{Testing Approach}

Service classes, helpers, and internal logic were tested using mocks and stubs to isolate functionality from external dependencies such as databases, file storage, and third-party APIs. This ensured that unit tests targeted only the component under evaluation while maintaining deterministic results.

\subsubsection{Failure and Boundary Validation}

Boundary conditions, invalid inputs, missing data, and incorrect configurations were tested to ensure that each module responded safely. Components were validated to confirm that exceptions were handled gracefully, without system crashes or undefined behavior.

% -------------------------------------------------------------
% ----------------------- TABLE 1 ------------------------------
% -------------------------------------------------------------

\subsubsection{Authentication and File Services Unit Tests}

\begin{figure}[H]
    \centering
    \includegraphics[width=0.90\textwidth]{images/wboxutest1.png}
    \caption{Unit Test Evidence 1}
    \label{fig:wboxutest1}
\end{figure}

\subsubsection{Search, Text Extraction, and Embedding Unit Tests}

\begin{figure}[H]
    \centering
    \includegraphics[width=0.90\textwidth]{images/wboxutest2.png}
    \caption{Unit Test Evidence 2}
    \label{fig:wboxutest2}
\end{figure}



\subsubsection{LLM, Editing, and Copilot Unit Tests}

\begin{figure}[H]
    \centering
    \includegraphics[width=0.90\textwidth]{images/wboxutest3.png}
    \caption{Unit Test Evidence 3}
    \label{fig:wboxutest3}
\end{figure}

\subsubsection{RAG Context, Entity Extraction, and Relationship Discovery Unit Tests}

\begin{figure}[H]
    \centering
    \includegraphics[width=0.90\textwidth]{images/wboxutest4.png}
    \caption{Unit Test Evidence 4}
    \label{fig:wboxutest4}
\end{figure}

\subsubsection{Unit Test Summary}

\begin{table}[h!]
\centering
\small
\begin{tabular}{|p{4cm}|p{3cm}|p{4cm}|}
\hline
\textbf{Category} & \textbf{Count} & \textbf{Notes} \\ \hline
Total Unit Tests Executed & 70 & Across all modules \\ \hline
Passing Tests & 64 &  \\ \hline
Failed Tests & 6 & Require fixes/mocking updates \\ \hline
Overall Pass Rate & 91\% & Stable with minor issues \\ \hline
\end{tabular}
\caption{Overall Unit Testing Summary}
\end{table}

\subsection{Test Evidence}

\begin{figure}[H]
    \centering
    \includegraphics[width=0.90\textwidth]{images/tests-summary.png}
    \caption{Test Execution Results}
\end{figure}


\section{Test Summary}
\label{sec:test-summary}

The testing results showed that the system successfully handled valid requests and appropriately rejected invalid or unauthorized access attempts. The core workflows such as project creation, file uploading, semantic search, and AI-assisted responses were verified to function as expected during the current iteration.

All critical defects identified during testing were resolved, and remaining enhancements will be addressed in the next development phase.

\subsection{Integration Testing}
\label{subsec:integration-testing}

\subsubsection{Summary}

Integration tests validated end-to-end behavior across API layers, databases, and services using the repository's pytest setup. The suite covers authentication, projects, files, search, chat, workflows, and health checks.

\begin{itemize}
    \item \textbf{Total integration tests:} ~60 (PHASE1 \& TESTING\_GUIDE) — reported as 100\% passing in the repo docs.
    \item \textbf{Pytest configuration:} Tests discovered under \texttt{backend/tests/}; notable files include \texttt{test\_auth\_api.py}, \texttt{test\_projects\_api.py}, \texttt{test\_files\_api.py}, \texttt{test\_search\_api.py}, \texttt{test\_chat\_api.py}, \texttt{test\_integration\_workflows.py}, and \texttt{test\_health.py}.
    \item \textbf{Notes:} Built from testing documentation (\texttt{WHITEBOX\_TESTING\_COMPLETE.md}, \texttt{PHASE1\_TESTING\_SUMMARY.md}, \texttt{TESTING\_GUIDE.md}, \texttt{TEST\_SUMMARY.md}) and concrete repo files (\texttt{conftest.py}, \texttt{pytest.ini}, run scripts). Exact counts may vary with updates.
\end{itemize}

\begin{figure}[H]
    \centering
    \includegraphics[width=0.90\textwidth]{images/apiintegtest1.png}
    \caption{Integration API Test Evidence}
    \label{fig:apiintegtest1}
\end{figure}

