\chapter{Project Requirements [AS PER FYP1 MID REPORT]}
This section outlines the necessary requirements for the successful completion of the project. Project requirements can be divided into two main categories: functional requirements, which describe the system’s core operations, and non-functional requirements, which specify performance, security, and usability standards.
\section{Use-case/Event Response Table/Storyboarding}
This section describes how users will interact with the system through various scenarios, often referred to as use cases or event responses.\\ A use case represents a specific interaction between the user and the system, detailing the steps involved and the expected system responses. Each use case typically includes key components such as a unique identifier, a description of the interaction, the user or system actor involved, preconditions, the main flow of events, and postconditions.\\ Additionally, storyboarding can be used as a visual aid to depict the sequence of actions, illustrating how the user progresses through the system step by step. Storyboards are helpful for capturing the user experience and ensuring that the design aligns with the intended functionality. \\

Depending on the project, use cases, storyboarding or Event-response table can be chosen as appropriate methodologies to illustrate user interactions. Examples of all the approaches are provided in Appendix A.



\section{Functional Requirements}
This section describes the functional requirements of the system expressed in the natural language style. This section is typically organized by feature as a system feature name and specific functional requirements associated with this feature. 

\subsection{Module 1}
Following are the requirements for module 1:
\begin{enumerate}
    \item FR1
    \item FR2
\end{enumerate}

\subsection{Module 2}
Following are the requirements for module 2:
\begin{enumerate}
    \item FR1
    \item FR2
\end{enumerate}

\section{Non-Functional Requirements}
This section specifies nonfunctional requirements. These quality requirements should be specific, quantitative, and verifiable. The following are some examples of documenting guidelines.

\subsection{Usability}
Usability requirements deal with ease of learning, ease of use, error avoidance and recovery, the efficiency of interactions, and accessibility. The usability requirements specified here will help the user interface designer create the optimum user experience.
Example:\\

USE-1: The COS shall allow a user to retrieve the previous meal ordered with a single interaction.

\subsection{Performance}
State specific performance requirements for various system operations. If different functional requirements or features have different performance requirements, it’s appropriate to specify those performance goals right with the corresponding functional requirements, rather than collecting them in this section. 
Example:\\

PER-1:	 95\% of webpages generated by the COS shall download completely within 4 seconds from the time the user requests the page over a 20 Mbps or faster Internet connection.



