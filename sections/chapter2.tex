\chapter{Project Requirements}
This chapter defines the requirements essential for developing CoWriteIA. These requirements were derived from the FYP-1 Proposal, Mid Report, and the SRS document. They describe how the system should behave, how users will interact with it, and what constraints and quality standards must be followed. The requirements are divided into functional and non-functional categories, with additional details about expected user interactions.

\section{Use-case}

CoWriteIA provides writers with AI-assisted features such as project indexing, semantic retrieval, character management, context-aware writing, and dialogue generation. To clearly define how users interact with these features, system behavior is modeled through detailed use cases.

A use case describes the interaction between the primary actor (the writer) and the system to accomplish a specific task. Each use case outlines the trigger event, preconditions, main workflow steps, and system responses. These models help ensure that system functionality aligns with user expectations and guide the design of interactive features.

In addition to textual use cases, a visual representation in the form of a Use-Case Diagram provides an overview of the major user interactions with CoWriteIA. This includes actions such as uploading documents, retrieving semantic context, generating content with the writing assistant, managing characters, and initiating dialogue generation. Such diagrams help maintain clarity, consistency, and completeness throughout the system design.

A \textbf{detailed use case} example for the core functionality \textbf{Generate Context-Aware Writing (UC-01)} is provided in Appendix~D. This detailed use case outlines the complete interaction flow including preconditions, main flow steps, alternate flows, exceptions, and postconditions. The use case demonstrates how the system retrieves semantic context, processes user requests through the AI model, and delivers generated content while handling various edge cases such as missing embeddings or service unavailability.

\section{Event--Response Table}

The following table summarizes how CoWriteIA responds to key user-initiated events during writing, project management, and context-aware content generation.

\begin{table}[H]
\centering
\renewcommand{\arraystretch}{1.3}
\begin{tabular}{p{4.2cm} p{10cm}}
\hline
\textbf{Event} & \textbf{System Response} \\
\hline

User uploads a document &
System extracts text, generates embeddings, indexes the content in the vector database, and updates the project repository. \\

User requests semantic search &
System retrieves top relevant embeddings, processes them through semantic ranking, and returns contextually relevant text segments. \\

User requests context-aware writing assistance &
System gathers project context, sends the prompt + context to the AI model, generates aligned text, and displays it in the writing panel. \\

User selects a character for dialogue generation &
System fetches character traits and stored personality data, and generates dialogue consistent with the character’s profile. \\

User updates character information &
System stores the updated traits, recalculates related embeddings, and refreshes character-linked semantic memory. \\

User performs research query &
System connects to external sources (via integrated APIs), retrieves factual data, summarizes it, and provides sources if available. \\

User initiates gap analysis &
System scans the document structure, identifies missing or weak sections, highlights inconsistencies, and suggests improvements. \\

User edits generated content &
System updates the project document, regenerates relevant embeddings, and reindexes modified sections. \\

User views project overview &
System displays indexed chapters, character profiles, notes, semantic connections, and recent AI-generated outputs. \\

AI model becomes unavailable &
System displays an error message, logs the failure, and advises the user to retry once service availability is restored. \\
\hline
\end{tabular}
\caption{Event--Response Table for CoWriteIA}
\end{table}



\section{Functional Requirements}
The functional requirements describe the operations the CoWriteIA system must support. These requirements come directly from the system features identified in the Proposal and SRS.

\subsection{Module 1: Project Indexing and Semantic Memory}
This module handles ingestion, indexing, and semantic storage of project documents.
\begin{enumerate}
    \item The system shall allow users to upload project files including text documents, chapters, notes, and research material.
    \item The system shall extract, segment, and convert uploaded content into embeddings for semantic retrieval.
    \item The system shall store embeddings in a vector database.
    \item The system shall provide semantic search capabilities based on meaning rather than keyword matching.
    \item The system shall return ranked search results with source references.
\end{enumerate}

\subsection{Module 2: Context-Aware Writing Assistant}
This module generates content aligned with user writing style and project context.
\begin{enumerate}
    \item The system shall analyze previous project content to understand tone, terminology, and style.
    \item The system shall allow users to generate context-aware text suggestions based on previous chapters or notes.
    \item The system shall retrieve relevant context automatically when generating new content.
    \item The system shall maintain style consistency between newly generated and existing content.
\end{enumerate}

\subsection{Module 3: Character and Scene Management}
This module manages characters, their traits, and related narrative structures.
\begin{enumerate}
    \item The system shall allow users to create, edit, and store character profiles.
    \item The system shall store attributes such as personality, relationships, behaviors, and backstory.
    \item The system shall allow users to manage scenes and attach characters to scenes.
    \item The system shall assist in retrieving character information when generating story text or dialogue.
\end{enumerate}

\subsection{Module 4: Dialogue Generation}
This module generates consistent, character-matching dialogue.
\begin{enumerate}
    \item The system shall generate dialogue aligned with character personality and tone.
    \item The system shall allow users to request dialogue for specific characters or scenes.
    \item The system shall ensure continuity between generated dialogue and existing narrative.
\end{enumerate}

\subsection{Module 5: Research Integration Module}
This module retrieves factual information to support realistic writing.
\begin{enumerate}
    \item The system shall allow users to search for factual references.
    \item The system shall fetch external information from trusted research sources.
    \item The system shall present research results with citations.
\end{enumerate}

\subsection{Module 6: Project Query Interface}
This module allows users to ask natural-language questions about their project.
\begin{enumerate}
    \item The system shall process user queries related to characters, scenes, chapters, or events.
    \item The system shall fetch relevant information from the semantic memory.
    \item The system shall provide answers with linked source passages.
\end{enumerate}

\subsection{Module 7: Gap Analysis Module}
This module identifies missing or weak areas of the writer’s draft.
\begin{enumerate}
    \item The system shall analyze uploaded chapters for missing elements such as incomplete scenes or inconsistent character behavior.
    \item The system shall highlight areas needing expansion or clarification.
    \item The system shall provide suggestions to improve narrative flow and completeness.
\end{enumerate}

\section{Non-Functional Requirements}
Non-functional requirements ensure that CoWriteIA performs reliably, efficiently, and securely.

\subsection{Usability}
\begin{enumerate}
    \item The system shall provide a simple and intuitive interface accessible to writers with minimal technical expertise.
    \item The system shall allow users to perform core actions such as uploading files, searching, and generating text within no more than three interactions.
    \item The UI shall clearly present semantic search results with source references.
\end{enumerate}

\subsection{Performance}
\begin{enumerate}
    \item The system shall index uploaded documents within 5 seconds for an average chapter-length file.
    \item Semantic search results shall appear within 2 seconds of a query.
    \item Generated text responses shall be produced within 3–5 seconds depending on context length.
\end{enumerate}

\subsection{Security}
\begin{enumerate}
    \item User project data shall be stored securely using encrypted connections.
    \item Only authenticated users shall access their own documents.
    \item The system shall not use project data for training without user permission.
\end{enumerate}

\subsection{Reliability}
\begin{enumerate}
    \item The system shall maintain uptime of at least 99% during active usage periods.
    \item The system shall recover gracefully from API or model failures by retrying or presenting fallback responses.
\end{enumerate}

\subsection{Maintainability}
\begin{enumerate}
    \item The codebase shall follow modular architecture to allow updates to individual agents.
    \item The system shall support integration of new language models without requiring major structural changes.
\end{enumerate}

\subsection{Compatibility}
\begin{enumerate}
    \item The system shall run on modern browsers including Chrome, Edge, and Firefox.
    \item The frontend shall be built using Next.js and shall be compatible with desktop and tablet interfaces.
\end{enumerate}

\subsection{Scalability}
\begin{enumerate}
    \item The vector database shall support scaling as the user creates larger projects.
    \item The system shall handle multiple concurrent requests without significant performance degradation.
\end{enumerate}
