\chapter{Introduction [AS PER SCOPE DOCUMENT]}
\label{sec:introduction}
The "Introduction" section sets the foundation for the entire report and should be carefully crafted to give the reader a clear understanding of the project’s scope, objectives, and context. Here’s what you can include:

Background:

Begin by providing background information on the broader topic or field your development project is related to.
Highlight any existing problems, gaps, or opportunities that led to the motivation for your project.
Include relevant references to previous research, development projects, or technologies. For example, you might want to cite studies or prior systems that failed to address the issues you're focusing on. \cite{test}.


\section{Existing Solutions}

For the Existing Solutions section, students should research and describe various solutions, methods, or approaches that have been previously developed to solve the problem they are addressing. This section should include an overview of the different techniques, technologies, or models that are currently in use or have been applied in the past. Students should critically assess the strengths and weaknesses of each solution, highlighting any gaps or limitations that their project aims to address. Additionally, this section should show an understanding of the state-of-the-art in the field and provide context for why the proposed solution is necessary or an improvement over existing ones.


\begin{table}[H]
\centering
\caption{Comparison of Existing Solutions}
\begin{tabular}{|c|p{5cm}|p{5cm}|}
\hline
\textbf{System Name}   & \textbf{System Overview} & \textbf{System Limitations} \\ \hline
System 1               & Brief description of what System 1 does and its primary features. & Highlight the shortcomings or issues with System 1. \\ \hline
System 2               & Brief description of what System 2 does and its primary features. & Highlight the shortcomings or issues with System 2. \\ \hline
System 3               & Brief description of what System 3 does and its primary features. & Highlight the shortcomings or issues with System 3. \\ \hline
\end{tabular}
\label{tab:system_comparison}
\end{table}

\section{Problem Statement}
Clearly define the specific problem your project aims to solve. Explain why this problem is important and how solving it will contribute to the field. This section is critical as it sets the context for your work.
This can include technical challenges, system inefficiencies, or user experience problems.
Why you are developing this system?
What problem does your software solve?
(Usually in 10-16 sentences)

\textbf{Example:}

\textit{In recent years, the surge in online shopping has resulted in a significant increase in the demand for e-commerce platforms. However, most small to medium-sized businesses struggle to adopt sophisticated e-commerce systems due to high costs, complexity, and lack of customization. Current available solutions either require substantial financial investment or offer limited functionality, leading to inefficiencies in managing product inventories, processing orders, and handling customer interactions.}

\textit{This project aims to address this issue by developing a cost-effective, customizable, and user-friendly e-commerce platform tailored specifically for small to medium-sized enterprises (SMEs). The platform will integrate key functionalities such as inventory management, order tracking, and customer relationship management into a single, unified system that can be easily deployed with minimal technical expertise.}

\section{Scope}
The scope of this project defines what will be included and excluded, helping to set clear expectations for the reader. This section outlines the boundaries and limitations of the development project.

This can include:
\begin{itemize}
    \item What the project will cover: You define the main features and functionality that the project will focus on, including the specific tools, systems, and technologies to be developed.
    \item What the project will not cover: This outlines the limitations, clarifying any areas that the project will not address (e.g., advanced features or post-deployment support). 

\end{itemize}
(Usually in 14-18 sentences)

\section{Modules}
This section describes the key functional units or modules of the project, each designed to perform a specific task. The breakdown of the project into modules ensures that each part of the system is manageable and addresses specific objectives. Each module should be described in terms of its functionality and how it interacts with other modules in the system. \\

The modules should be logically organized to reflect the overall system workflow, ensuring that each module description is comprehensive but concise.
\subsection{Module 1}
[2-4 lines description of the module]
\begin{enumerate}
    \item Feature 1
    \item Feature 2
\end{enumerate}
\subsection{Module 2}
[2-4 lines description of the module]
\begin{enumerate}
    \item Feature 1
    \item Feature 2
\end{enumerate}

\subsection{Example:}
\subsubsection{Inventory Management Module}
This module focuses on inventory management, allowing businesses to efficiently track and manage their product stock levels. It integrates features for adding new products, updating stock information, and generating low-stock alerts.

\begin{enumerate}
    \item Ability to add, update, and remove products from inventory.
    \item Automated alerts for low stock levels to aid in restocking decisions.
\end{enumerate}

\section{Work Division}

