\chapter{Introduction}
\label{sec:introduction}

CoWriteIA is an AI-powered writing assistant designed to support creative writers, novelists, researchers, and content creators. Modern writing workflows require managing notes, drafts, characters, scenes, and references across several disconnected tools, which often breaks focus and reduces productivity. Writers struggle to maintain narrative consistency, track earlier ideas, and keep a coherent writing style when working on long projects. Existing tools offer only isolated features such as grammar correction or basic generation and do not provide project-level understanding or semantic memory~\cite{lewis2020rag,reimers2019sentencebert}. 

CoWriteIA addresses these issues by creating a unified, intelligent workspace where all project files are indexed, searchable, and semantically connected. By integrating project-level memory, context-aware writing, dialogue generation, character management, and research support, the system helps writers maintain consistency and improve their creative process. The platform is designed to reduce cognitive load, avoid fragmented workflows, and provide meaningful AI assistance throughout the writing journey. 
This chapter presents the background, existing solutions, problem statement, scope, project modules, and work division that form the foundation of this system.
 
\section{Existing Solutions}

Several writing and AI-assisted tools exist, but each focuses on limited aspects of the writing process. Grammarly offers grammar correction and style suggestions but lacks deep contextual awareness across long projects. Notion AI and Jasper AI provide generative assistance but do not maintain story-level continuity or user-specific writing style. Tools like Scrivener help with organization but have no semantic understanding or AI memory. As a result, writers must repeatedly switch between applications to manage notes, drafts, characters, and research, leading to inefficiency and inconsistent writing flow.

\begin{table}[H]
\centering
\caption{Comparison of Existing Solutions}
\begin{tabular}{|c|p{5cm}|p{5cm}|}
\hline
\textbf{System Name} & \textbf{System Overview} & \textbf{System Limitations} \\ \hline
Grammarly & Provides grammar correction, clarity improvements, and tone suggestions. & No project awareness, no memory of earlier chapters, no semantic search. \\ \hline
Notion AI & Offers AI-assisted content generation and note organization. & Cannot maintain narrative consistency; lacks dialogue and character support. \\ \hline
Scrivener & Strong organizational tool for large writing projects with chapter/scene structure. & No AI support, no semantic retrieval, no automated gap or style analysis. \\ \hline
\end{tabular}
\label{tab:system_comparison}
\end{table}

\section{Problem Statement}

Writers working on long-form projects often lose track of earlier ideas, character traits, plot points, and stylistic decisions. This leads to inconsistencies, repeated ideas, and a break in narrative flow. Existing tools either provide isolated writing support or document organization but do not combine both with meaningful context. AI tools can generate text but fail to maintain project-level continuity, making the generated text feel disconnected from the writer's established style or storyline. Writers spend significant time searching through older drafts to recall information~\cite{guu2020realm,lewis2020rag}.

CoWriteIA aims to solve these problems by offering an intelligent system that continuously indexes all project content, retrieves relevant context, and assists writers in generating text that aligns with their established narrative and writing style. By maintaining a unified knowledge base and supporting character consistency, scene management, dialogue generation, and semantic search, the system reduces cognitive overhead and improves creative flow.

\section{Scope}

The scope of \textbf{CoWriteIA} encompasses the development of an intelligent, AI-driven writing environment designed to support long-form creative and research-oriented projects. The system focuses on providing project-level understanding, semantic reasoning, and context-aware assistance to help writers maintain consistency, creativity, and workflow continuity.

\subsection*{In-Scope Features}

\begin{itemize}
    \item \textbf{Project Management and Indexing} — Upload, organize, and store documents across multiple writing projects.
    
    \item \textbf{Semantic Search and Retrieval} — Retrieve information using a vector-based semantic memory instead of keyword matching~\cite{reimers2019sentencebert,khattab2020colbert}.
    
    \item \textbf{Context-Aware Writing Assistance} — Generate or refine text that aligns with prior chapters, scenes, tone, and writing style.
    
    \item \textbf{Dialogue and Narrative Support} — Produce character-consistent dialogues and maintain scene-level continuity.
    
    \item \textbf{Character and Knowledge Storage} — Maintain detailed character profiles, relationships, and narrative facts.
    
    \item \textbf{Research Integration} — Fetch, summarize, and integrate external factual information to support realistic writing.
    
    \item \textbf{Style Adaptation} — Adapt AI-generated content to match the user’s writing voice across the project.
\end{itemize}

\subsection*{Out-of-Scope Features}

\begin{itemize}
    \item Plagiarism detection or originality scoring.
    \item Multimedia editing such as images, audio, or video.
    \item Complete publishing or formatting pipelines (e.g., EPUB, print layout).
    \item Full-fledged grammar-only tools or scriptwriting-specific toolkits.
\end{itemize}

\subsection*{Scope Summary}

CoWriteIA is limited to enhancing the \emph{writing process itself}: maintaining narrative consistency, providing intelligent context-aware support, and simplifying research-driven creative decision making. The system does not aim to replace editing, publishing, or multimedia tools, but instead acts as a unified, intelligent workspace to elevate author productivity and coherence throughout the creative workflow.

\section{Modules}

The project consists of several modules, each responsible for a unique part of the writing workflow.

\subsection{Module 1: Project Indexing and Semantic Memory}
This module extracts, embeds, and organizes all project content into a semantic database for context-aware retrieval~\cite{reimers2019sentencebert}. 
\begin{enumerate}
    \item Automatic project indexing and embedding generation.
    \item Semantic search based on meaning instead of keywords.
\end{enumerate}

\subsection{Module 2: Context-Aware Writing Assistant}
This module provides intelligent writing suggestions that match the user's tone and project context~\cite{lewis2020rag}.
\begin{enumerate}
    \item Generates coherent drafts aligned with past content.
    \item Retrieves relevant information to maintain consistency.
\end{enumerate}

\subsection{Module 3: Character and Scene Management}
This module stores and manages characters, scenes, and narrative details. 
\begin{enumerate}
    \item Character profiles with traits and relationships.
    \item Scene storage and tracking.
\end{enumerate}

\subsection{Module 4: Dialogue Generation}
Generates natural, character-consistent dialogue suggestions. 
\begin{enumerate}
    \item Dialogue generation based on personality and context.
    \item Supports narrative flow within scenes.
\end{enumerate}

\subsection{Module 5: Research Integration}
Fetches factual data from external sources for realistic writing. 
\begin{enumerate}
    \item Web-based research retrieval.
    \item Insertable factual references.
\end{enumerate}

\subsection{Module 6: Project Query Interface}
Allows users to ask natural-language questions about their own project. 
\begin{enumerate}
    \item Semantic question-answering.
    \item Source-linked responses.
\end{enumerate}

\subsection{Module 7: Gap Analysis Module}
Analyzes draft content to find missing or weak areas. 
\begin{enumerate}
    \item Identifies incomplete sections.
    \item Suggests improvements for clarity and consistency.
\end{enumerate}

\clearpage

\section{Work Division}

The work completed during FYP-1 was divided into two major iterations.  
A summary of responsibilities is presented in the tables below.

\subsection*{Iteration I Tasks}

\begin{table}[!ht]
\centering
\caption{Work Division for Iteration I}
\begin{tabular}{lccc}
\toprule
\textbf{Task} & \textbf{Ayesha} & \textbf{Junaid} & \textbf{Aisha} \\
\midrule
SRS Document & \checkmark & \checkmark & \checkmark \\
UML Diagrams & \checkmark & \checkmark & \checkmark \\
UI Design &  & \checkmark & \checkmark \\
Database Connection &  & \checkmark & \checkmark \\
Frontend (Main Pages) & \checkmark &  &  \\
Project Indexing Agent & \checkmark & \checkmark &  \\
Knowledge Retrieval Agent &  & \checkmark & \checkmark \\
ChatBot Support &  & \checkmark  &  \\
\bottomrule
\end{tabular}
\end{table}
\FloatBarrier

\subsection*{Iteration II Tasks}

\begin{table}[!ht]
\centering
\caption{Work Division for Iteration II}
\begin{tabular}{lccc}
\toprule
\textbf{Task} & \textbf{Ayesha} & \textbf{Junaid} & \textbf{Aisha} \\
\midrule
Inline Copilot Support & \checkmark &  &  \\
Context-Aware Writing Module & \checkmark &  &  \\
Gap Analysis Module &  &  & \checkmark \\
Style Adaptation &  & \checkmark &  \\
Testing & \checkmark & \checkmark & \checkmark \\
Documentation (All Sections) & \checkmark & \checkmark & \checkmark \\
\bottomrule
\end{tabular}
\end{table}
\FloatBarrier

\subsection*{Project Timeline}

\begin{figure}[H]
    \centering
    \includegraphics[width=\textwidth]{images/gantt-chart.png}
    \caption{Gantt Chart - Project Timeline}
    \label{fig:gantt}
\end{figure}
\FloatBarrier
\clearpage

