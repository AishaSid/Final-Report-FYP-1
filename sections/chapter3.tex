\chapter{System Overview}

\subsection*{Introduction}
This chapter provides a high-level description of the system's overall functionality, context, and architectural design. The aim is to present how the major parts of the system interact and why the system has been decomposed into specific modules and tiers.

\subsection*{System Overview}
\textbf{CoWriteIA} is an AI-assisted writing platform designed to support writers through various stages of the creative process~\cite{notionai,grammarly,scrivener}. The system enables users to:

\begin{itemize}
    \item Create and manage writing projects
    \item Upload and organize documents
    \item Manage character profiles and story elements
    \item Generate and refine written content using AI assistance
\end{itemize}

To efficiently support these features, the system separates concerns into distinct, coordinated components: user interaction, application logic, AI processing, and data management.

\section{Architectural Design}

The architecture of CoWriteIA follows a \textbf{multi-tier model}, combining \textbf{client–server} and \textbf{layered} principles. This structure organizes the system into four logical tiers, each with specific responsibilities to ensure maintainability, scalability, and clarity.

\subsection*{Architectural Tiers}
\begin{table}[htbp]
\centering
\renewcommand{\arraystretch}{1.4}
\begin{tabular}{@{} l l p{0.5\textwidth} @{}}
\toprule
\textbf{Tier} & \textbf{Component} & \textbf{Responsibilities} \\
\midrule
\textbf{Presentation} & Frontend Application & User interaction, interface rendering, and user input handling \\
\midrule
\textbf{Application Logic} & Backend (API Server) & Authentication, business workflows, and project/document/character management \\
\midrule
\textbf{AI Processing} & Processing Worker & Embeddings generation, semantic retrieval, and AI text generation \\
\midrule
\textbf{Data} & Storage Systems & Structured data, vector storage, and file management \\
\bottomrule
\end{tabular}
\caption{Architectural Tiers and Responsibilities}
\label{tab:tiers}
\end{table}

\subsection*{Data Storage Components}
\begin{itemize}
    \item \textbf{Main Database}: Stores structured system data (users, projects, characters)
    \item \textbf{Vector Database}: Manages embeddings for semantic retrieval and search
    \item \textbf{File Storage}: Handles uploaded documents and exported files
\end{itemize}

\subsection*{Architectural Diagram}
\begin{figure}[htbp]
    \centering
    \includegraphics[width=0.85\textwidth]{images/Box-line.png}
    \caption{Box and Line Diagram}
    \label{fig:Box-line}
\end{figure}

A detailed architecture diagram illustrating the complete multi-tier structure of CoWriteIA is provided in Appendix C (Figure~\ref{fig:cowriteia-architecture}).

\subsection*{Diagram Development Process}
\begin{itemize}
    \item \textbf{Initial Design Stage}: Create a \textit{Box and Line Diagram} for simpler representation of systems
    \item \textbf{Finalization Stage}: After selecting the architecture style/pattern (MVC, Client-Server, Layered, Multi-tiered), create detailed mapping of modules/components to each part of the architecture
\end{itemize}

\subsection*{Design Principles}
This architectural decomposition supports the system's functional requirements while ensuring:

\begin{itemize}
    \item \textcolor{blue}{\textbf{Flexibility}}: Components can be modified independently
    \item \textcolor{blue}{\textbf{Maintainability}}: Clear separation of concerns simplifies updates and debugging
    \item \textcolor{blue}{\textbf{Scalability}}: Each tier can be scaled independently based on demand
    \item \textcolor{blue}{\textbf{Security}}: Controlled data flow between tiers with proper authentication
\end{itemize}

\subsection*{Key Architectural Decisions}
\begin{itemize}
    \item \textbf{Separation of AI Processing}: Intensive AI tasks are handled by dedicated workers to maintain API responsiveness
    \item \textbf{Multi-tier Structure}: Enables independent development, testing, and deployment of each tier
    \item \textbf{Modular Design}: Supports future enhancements and feature additions
\end{itemize}

\section{Data Design}

The data design of CoWriteIA transforms functional requirements into structured data models that support \textbf{project creation}, \textbf{AI-assisted writing}, \textbf{semantic search}, \textbf{character management}, and \textbf{dialogue modelling}.

\subsection*{Data Organization}

The system organizes data into three primary categories:

\begin{itemize}
    \item \textbf{Relational Data}: User accounts, projects, documents, characters, and chat sessions stored in a structured database with defined schemas and foreign key relationships
    \item \textbf{Vector Data}: High-dimensional embeddings generated from text content, enabling semantic similarity search and context retrieval
    \item \textbf{Binary Data}: Uploaded files and exported documents stored separately with metadata linkage
\end{itemize}

\subsection*{Data Flow and Transformation}

Data flows through multiple transformation stages:

\begin{itemize}
    \item \textbf{Input Processing}: User actions validate and create/update structured database records
    \item \textbf{Embedding Generation}: Text content is transformed into vector representations for AI operations
    \item \textbf{Retrieval Augmentation}: Semantic queries fetch relevant context from vector and relational stores
    \item \textbf{Output Generation}: AI responses and exports combine retrieved data with generated content
\end{itemize}

This design ensures \textcolor{blue}{\textbf{data integrity}}, \textcolor{blue}{\textbf{efficient retrieval}}, and \textcolor{blue}{\textbf{scalable processing}}. Recent work on efficient local models such as Phi-3 highlights practical on-device capabilities for embedding and retrieval~\cite{phi3}.

Figure~\ref{fig:erd} provides the Entity–Relationship Diagram showing database schema structure, while Figure~\ref{fig:domainmodel} presents the conceptual domain model.

\FloatBarrier
\begin{figure}[htbp]
    \centering
    \includegraphics[width=\textwidth]{images/erd.png}
    \caption{Entity–Relationship Diagram for CoWriteIA}
    \label{fig:erd}
\end{figure}

\FloatBarrier

\subsection{Class Diagram}

The class diagram represents the object-oriented view of the system and shows classes, attributes, and associations used by the application logic. It provides a structural foundation for backend implementation and helps ensure consistency between the conceptual design and the code-level architecture.

\begin{figure}[htbp]
    \centering
    \includegraphics[width=\textwidth]{images/classdiagram.png}
    \caption{Class Diagram for CoWriteIA}
    \label{fig:classdiagram}
\end{figure}



\section{Domain Model}

The domain model provides a conceptual view of the system's core entities and their relationships, bridging functional requirements and technical implementation. It describes the key business objects, their attributes, and how they interact within CoWriteIA.

\subsection*{Entities and Attributes}

The major entities of the system include:

\begin{itemize}
    \item \textbf{User}: Represents a system user who owns projects and interacts with the AI. Attributes include \texttt{user\_id}, \texttt{name}, \texttt{email}, \texttt{date\_of\_birth}, \texttt{password\_hash}, \texttt{role}, and \texttt{created\_at}.
    
    \item \textbf{Project}: A workspace created by the user. Attributes include \texttt{project\_id}, \texttt{title}, \texttt{created\_at}, and \texttt{updated\_at}.
    
    \item \textbf{Document}: Represents a chapter or section of a project. Attributes include \texttt{document\_id}, \texttt{title}, \texttt{order\_num}, \texttt{created\_at}, and \texttt{updated\_at}.
    
    \item \textbf{Character Repository}: A user-owned collection that groups all character profiles.
    
    \item \textbf{Character}: Represents a story character. Attributes include \texttt{character\_id}, \texttt{name}, \texttt{traits}, \texttt{history}, \texttt{created\_at}, and \texttt{updated\_at}.
    
    \item \textbf{ProjectCharacter}: An association class linking characters with specific projects. Stores project-specific values such as \texttt{age} and \texttt{events}.
    
    \item \textbf{Dialogue}: Represents character dialogues. Attributes include \texttt{dialogue\_id}, \texttt{content}, \texttt{context}, and \texttt{emotion\_tag}.
    
    \item \textbf{ProjectDialogue}: A linking entity that associates dialogues with a project.
    
    \item \textbf{Chat Session}: Represents an AI chat interaction initiated by the user. Identified by \texttt{chat\_id}.
    
    \item \textbf{Message}: Stores an individual message in a chat session. Attributes include \texttt{message\_id}, \texttt{sent\_by}, \texttt{timestamp}, and \texttt{content}.
    
    \item \textbf{Search Engine}: Represents the semantic search component.
    
    \item \textbf{Search Result}: Stores contextual search outputs. Attributes include \texttt{result\_id}, \texttt{query}, and \texttt{content}.
    
    \item \textbf{Knowledge Base}: Stores vector embeddings and source references. Attributes include \texttt{embedding\_id}, \texttt{vector\_representation}, and \texttt{source\_ids}.
    
    \item \textbf{Exported Project}: Represents generated output files such as PDF or Word exports.
\end{itemize}

\subsection{Relationships and Associations}

The domain model defines the relationships that structure how information flows across the system:

\begin{itemize}
    \item A \textbf{User} owns multiple \textbf{Projects}.
    \item A \textbf{Project} contains multiple \textbf{Documents}.
    \item A \textbf{User} owns a \textbf{Character Repository}, which stores many \textbf{Characters}.
    \item A \textbf{Project} includes multiple \textbf{Characters} via \textbf{ProjectCharacter}.
    \item A \textbf{Character} speaks one or more \textbf{Dialogues}.
    \item A \textbf{Project} associates specific dialogues through \textbf{ProjectDialogue}.
    \item A \textbf{User} initiates a \textbf{Chat Session}.
    \item A \textbf{Chat Session} stores multiple \textbf{Messages}.
    \item The \textbf{LLM Chatbot} interacts with the \textbf{Semantic Search Engine} and \textbf{Knowledge Base} to generate responses.
    \item The \textbf{Search Engine} produces multiple \textbf{Search Results}, which may be injected into messages or documents.
\end{itemize}

These relationships ensure that writing, character creation, semantic search, and AI interactions remain consistent and interconnected across the system.

\subsection{Domain Model Diagram}

Figure~\ref{fig:domainmodel} presents the complete domain model for CoWriteIA. It includes all major entities, attributes, and associations, including weak entities and conceptual relationships.

\begin{figure}[htbp]
    \centering
    \includegraphics[width=\textwidth]{images/domain.png}
    \caption{Domain Model Diagram for CoWriteIA}
    \label{fig:domainmodel}
\end{figure}

\FloatBarrier

\section{Design Models}

\subsection{Sequence Diagrams}
This subsection presents sequence diagrams for the four main system flows, based on the implementation logic described in Chapter 4. Each diagram outlines the interactions between system components for a key workflow.

\subsubsection*{1. File Upload \& Chunking}
	\textbf{Actors:} User, API Endpoint, TextExtractionService, TextChunkRepository

    \begin{enumerate}
        \item User uploads file via API endpoint.
        \item API calls \texttt{TextExtractionService.extract\_text\_from\_file}.
        \item API calls \texttt{TextExtractionService.chunk\_text} to split text into chunks.
        \item Chunks are created and saved via \texttt{TextChunkRepository}.
    \end{enumerate}

    \begin{figure}[htbp]
        \centering
        \includegraphics[width=\textwidth]{images/seqdiag_FileUploadAndChunking.png}
        \caption{Sequence Diagram: File Upload \& Chunking}
        \label{fig:seq-upload-chunk-image}
    \end{figure}
    \FloatBarrier

\subsubsection*{2. Embedding Generation \& Storage}
    \textbf{Actors:} API/Orchestrator, EmbeddingService, TextChunkRepository, ChromaService

    \begin{enumerate}
        \item \textbf{Batch Encoding:} The system extracts content strings from the chunks and calls \texttt{EmbeddingService.generate\_embeddings}. This runs a threadpool execution where the model loads lazily to convert text into dense vector representations.
        \item \textbf{Metadata Persistence:} The system iterates through the original chunks and generated embeddings simultaneously. For each pair, a \texttt{TextChunk} model is built and saved to MongoDB via the \texttt{TextChunkRepository}.
        \item \textbf{Vector Indexing:} If the \texttt{ChromaService} is configured and available:
        \begin{itemize}
            \item The system maps the saved chunks to Chroma-compatible entries (ID, vector, and metadata).
            \item It performs a bulk insert via \texttt{ChromaService.add\_embeddings} to enable semantic search.
        \end{itemize}
        \item \textbf{Fallback:} If Chroma is unavailable, the embeddings remain stored solely within the MongoDB documents.
    \end{enumerate}

    \begin{figure}[htbp]
        \centering
        \includegraphics[width=\textwidth]{images/seqdiag_EmbeddingGenerationAndStorage.png}
        \caption{Sequence Diagram: Embedding Generation \& Storage}
        \label{fig:seqdiag_embedding_generation_storage}
    \end{figure}
    \FloatBarrier

\subsubsection*{3. Semantic Search / RAG Context Retrieval}
    \textbf{Actors:} User, API Endpoint, RAGContextService, ChromaService, SearchService, TextChunkRepository

    \begin{enumerate}
        \item User sends a search/query.
        \item API calls \texttt{RAGContextService.assemble\_context}.
        \item RAGContextService generates query embedding.
        \item ChromaService or TextChunkRepository performs vector search.
        \item Top chunks and metadata are assembled and returned.
    \end{enumerate}

    \begin{figure}[htbp]
        \centering
        \includegraphics[width=\textwidth]{images/seqdiag_SemanticSearchRAGContextRetrieval.png}
        \caption{Sequence Diagram: Semantic Search / RAG Context Retrieval}
        \label{fig:seqdiag_semantic_rag}
    \end{figure}
    \FloatBarrier

\subsubsection*{4. AI Response Generation}
    \textbf{Actors:} User, API Endpoint, RAGContextService, AIChatService, GroqService, MessageRepository

    \begin{enumerate}
        \item User sends a message.
        \item API calls \texttt{RAGContextService.assemble\_context}.
        \item Context is formatted for LLM.
        \item AIChatService builds prompt and calls GroqService (LLM).
        \item LLM response is cleaned and saved via MessageRepository.
        \item Response returned to user.
    \end{enumerate}

    \begin{figure}[htbp]
        \centering
        \includegraphics[width=\textwidth]{images/seqdiag_AIResponseGeneration.png}
        \caption{Sequence Diagram: AI Response Generation}
        \label{fig:seqdiag_ai_response_generation}
    \end{figure}
    \FloatBarrier

\subsubsection*{5. Reprocessing \& Incremental Update}
    \textbf{Actors:} User/Admin, TextChunkRepository, FileRepository, TextExtractionService, EmbeddingService

    \begin{enumerate}
        \item File is marked for reprocessing.
        \item Existing chunks are deleted.
        \item File is marked ready for processing.
        \item Text is extracted and chunked again.
        \item Chunks are inserted.
        \item Embeddings are regenerated and stored.
    \end{enumerate}

    \begin{figure}[htbp]
        \centering
        \includegraphics[width=\textwidth]{images/seqdiag_ReprocessingAndIncrementalUpdate.png}
        \caption{Sequence Diagram: Reprocessing \& Incremental Update}
        \label{fig:seqdiag_reprocessing_incremental_update}
    \end{figure}
    \FloatBarrier

\subsection{State Transition Diagram}

Figure~\ref{fig:state} illustrates state transitions for key entities based on user actions and system events. The diagram shows how entities move between states:

\begin{itemize}
    \item \textbf{Project}: Transitions from Draft → Active → Archived/Deleted based on user actions
    \item \textbf{Document}: Moves through Creating → Editing → Saved → Publishing → Published states
    \item \textbf{Chat Session}: Cycles between Idle → Processing → Waiting for User → Completed
    \item \textbf{Character}: Changes from Draft → Active → Archived as needed
\end{itemize}

These transitions ensure proper lifecycle management and data consistency throughout the system.

\begin{figure}[htbp]
    \centering
    \includegraphics[width=1.1\textwidth]{images/state.png}
    \caption{State Transition Diagram for CoWriteIA}
    \label{fig:state}
\end{figure}

\FloatBarrier